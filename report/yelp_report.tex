%
% CS6220 Data Mining Project
%
\documentclass[12pt]{article}

%
% Packages
%
\usepackage{amsmath}
\usepackage{enumerate}
\usepackage[utf8]{inputenc}

\usepackage{clrscode3e}
\usepackage{tikz}

\RequirePackage{graphics}

\usepackage{graphicx}
\graphicspath{ {imgs/} }

%
% Document Settings
%
\setlength{\parskip}{1pc}
\setlength{\parindent}{0pt}
\setlength{\topmargin}{-3pc}
\setlength{\textheight}{9.5in}
\setlength{\oddsidemargin}{0pc}
\setlength{\evensidemargin}{0pc}
\setlength{\textwidth}{6.5in}

\title{Restaurant Recommendations using Yelp}
\author{Manoj Shinde, Tyler Brown, Shohit Bajaj}
\date{ }

% START DOCUMENT
\begin{document}

\maketitle

%\tableofcontents

\begin{abstract}
  When traveling or desiring to experience new food, a person will often seek out
  restaurant recommendations. The availiablity of data on restaurant preferences, via Yelp,
  creates an opportunity for someone to have a more personalized experience.
  We recommend restaurants to a user based on their past reviews and features which
  differentiate types and quality of restaurants. Supervised learning techniques allow
  us to predict the rating of unseen restaurants by each user. Restaurant rating
  predictions allow us to rank which dining experiences are most likely to be well
  received for each user.
  \end{abstract}

\section{Introduction}

There are many different types of restaurant experiences due to variation in
personal preferences. A service catering to these personal preferences will
generate value by improving customer convience
\cite{mackay_vandevijvere_xie_lee_swinburn_2017}. Yelp is a popular mobile/web
application that has tons of data on different businesses \cite{Restaura71:online}.
Consumer reviews have been shown to affect restaurant demand. A one-star increase in
Yelp ratings has been linked to a 5-9 percent increase in restaurant revenue
\cite{luca2016reviews}. Consumers do not use all available information and are more
responsive to quality changes that are more visible \cite{luca2016reviews}. It is
not practical for a consumer to go though every rating and review. Sorting through
each recommendation to see which is most relevant is also not practical for the
consumer. We use a supervised learning model to provide value by improving customer
convienence with restaurant recommendations powered by Yelp.

Restaurants on Yelp are given ratings using a five-star scale. We compare two
techniques to solve this multiclass problem. A multinomial logistic regression
is used as a baseline model. We then try to improve our accuracy with the use of
ensemble methods. Data from Yelp has been previously collected and made publicly
available on Kaggle \cite{YelpData59:online}. We demonstrate how to improve
consumer convienence by creating personalized restaurant recommendations with
Yelp data.

\section{Methodology}

Recommendation systems have historically been used in various other disciplines and
are considered quite effective in influencing the decision of a user. In this project
we are trying to make a recommendation based on the rating that a given user would give
to a specific restaurant.

For this project we have chosen Yelp’s dataset on Kaggle \cite{YelpData59:online}
which is 3GB in size. The dataset
has 5,200,000 user reviews, Information on 174,000 businesses, spanning 11 metropolitan
areas in four countries. The dataset has 7 csv files containing data on different
attributes of a business. Our focus is on yelp\_business.csv and yelp\_review.csv where
fields like user\_id, stars, review, category help us in establishing our model.

Multinomial logistic regression may consider features related to the type of restaurant
business, or individual characteristics about the user.

Ensemble methods may be able to bring in additional features. Location-based social
networks (LBSNs) are a combination of location-based networks and social media. There
may be geographic patterns of ratings of Yelp business venues within a city-wide
region \cite{sun2017spatial}. Including the AMOEBA algorithm may increase the
accuracy of our model. Additional programming may be required to include the AMOEBA
algorithm with the ensemble methods. We may include a classifier using deep
learning with the ensemble methods if there's time.

\section{Code}

Our code is being done in Python. The Multinomial Logisitic regression may be
implemented using Scikit-Learn or Statsmodels. 

The AMOEBA algorithm is a form of optimization problem. It is also known
as the Nelder-Mead method and available in Scipy \cite{Optimiza90:online}. Further
work will be necessary to see if spatial algorithms such as this can be
included as part of the ensemble methods. 

\section{Results}

Reviewing results here

\subsection{Multinomial Logistic Regression}

Here's what we found.

\subsection{Ensemble Methods}

Here's what else we found.


\section{Discussion}

We found that a certain model worked better. We found certain patterns
in the data, etc. We also have certain caveats related to the data that
we'll mention here.

\section{Future Work}

We may make a recommendation API for Yelp.

\section{Conclusion}

We found some things that worked well and other things that worked
less well. There seems to be a positive direction forward taking a
certain approach.


\bibliography{references} 
\bibliographystyle{ieeetr}

\end{document}
