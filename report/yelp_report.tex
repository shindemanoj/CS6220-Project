%
% CS6220 Data Mining Project
%
\documentclass[12pt]{article}

%
% Packages
%
\usepackage{amsmath}
\usepackage{enumerate}
\usepackage[utf8]{inputenc}

\usepackage{clrscode3e}
\usepackage{tikz}

\RequirePackage{graphics}

\usepackage{graphicx}
\graphicspath{ {imgs/} }

%
% Document Settings
%
\setlength{\parskip}{1pc}
\setlength{\parindent}{0pt}
\setlength{\topmargin}{-3pc}
\setlength{\textheight}{9.5in}
\setlength{\oddsidemargin}{0pc}
\setlength{\evensidemargin}{0pc}
\setlength{\textwidth}{6.5in}

\title{Restaurant Recommendations using Yelp}
\author{Manoj Shinde, Tyler Brown, Shohit Bajaj}
\date{ }

% START DOCUMENT
\begin{document}

\maketitle

%\tableofcontents

\begin{abstract}
  When traveling or desiring to experience new food, a person will often seek out
  restaurant recommendations. The availiablity of data on restaurant preferences, via Yelp,
  creates an opportunity for someone to have a more personalized experience.
  We recommend restaurants to a user based on their past reviews and features which
  differentiate types and quality of restaurants. Supervised learning techniques allow
  us to predict the rating of unseen restaurants by each user. Restaurant rating
  predictions allow us to rank which dining experiences are most likely to be well
  received for each user.
  \end{abstract}

\section{Introduction}

There are many different types of restaurant experiences due to variation in
personal preferences. A service catering to these personal preferences will
generate value by improving customer convience
\cite{mackay_vandevijvere_xie_lee_swinburn_2017}. Yelp is a popular mobile/web
application that has tons of data on different businesses \cite{Restaura71:online}.
Consumer reviews have been shown to affect restaurant demand. A one-star increase in
Yelp ratings has been linked to a 5-9 percent increase in restaurant revenue
\cite{luca2016reviews}. Consumers do not use all available information and are more
responsive to quality changes that are more visible \cite{luca2016reviews}. It is
not practical for a consumer to go though every rating and review. Sorting through
each recommendation to see which is most relevant is also not practical for the
consumer. We use a supervised learning model to provide value by improving customer
convienence with restaurant recommendations powered by Yelp.

Restaurants on Yelp are given ratings using a five-star scale. We compare two
techniques to solve this multiclass problem. A multinomial logistic regression
is used as a baseline model. We then try to improve our accuracy with the use of
ensemble methods. Data from Yelp has been previously collected and made publicly
available on Kaggle \cite{YelpData59:online}. We demonstrate how to improve
consumer convienence by creating personalized restaurant recommendations with
Yelp data.

Recommendation systems have historically been used in various other disciplines and
are considered quite effective in influencing the decision of a user. In this project
we are trying to make a recommendation based on the rating that a given user would give
to a specific restaurant. Yelp data \cite{YelpData60:online, YelpData59:online} has been
used in a variety of challenges and useful approaches have been well documented in a number of
academic articles \cite{yu2015restaurants, huang2014improving, fan2014predicting,asghar2016yelp}.
Given that the academic community has submitted a number of approaches to predicting
Yelp review ratings \cite{asghar2016yelp, potamias2012warm, fan2014predicting, yu2015restaurants},
we chose to replicate a paper and then extend their analysis. Yu et. al. \cite{yu2015restaurants}
investigate the dataset provided in Yelp Dataset Challenge Round 5 to predict a reviewers
next rating, given their rating history and other contextual factors. We replicate
Yu et. al. \cite{yu2015restaurants} and then extend their paper by developing a model which
generates output that is more consistent with what would be expected by a Yelp user.


\section{Methodology}

Our approach was to replicate results which Yu et. al. \cite{yu2015restaurants} found
fruitful and then extend their analysis with an emphasis on real world usability. To meet
these goals, we first conducted a data audit, completed a replication of fruitful results
from Yu et. al \cite{yu2015restaurants}, and then completed an extension of their results.

For this project we have chosen Yelp’s dataset on Kaggle \cite{YelpData59:online}
which is 3GB in size. The dataset
has 5,200,000 user reviews, Information on 174,000 businesses, spanning 11 metropolitan
areas in four countries. The dataset has 7 csv files containing data on different
attributes of a business. Our focus is on yelp\_business.csv and yelp\_review.csv where
fields like user\_id, stars, review, category help us in establishing our model.

\subsection{Data Audit}

Upon downloading our copy of Yelp Data hosted on Kaggle \cite{YelpData59:online}, we performed
an audit of the data to assess usability and any data quality issues. We found that our
dependent variable, Yelp review stars, contained imbalanced classes. There were considerably
more 4 and 5 star ratings than 1-3 star ratings. The Yelp dataset came to us in several tables.
We created an interactive data visualization to show the relationships between tables by creating
an undirected graph where each node was a table name and each edge showed a foreign key
relationship between nodes. Understanding the data schema was crucial to replicating
Yu et. al. \cite{yu2015restaurants} as well as thinking about and implementing new features.

The second phase of our data audit created tables showing the percentage of NA, blank, and
both NA and blank values in each attribute for each table. This excercise quickly showed us
which tables contained the most usable information for our purposes. For example, the Yelp
Business Attributes table contained several attributes which seemed like they would be
relevant features such as presence of Wifi. Upon conducting an assessment, we found that
the Yelp Business Attributes was almost exclusively populated by NA values. By contrast, the
Yelp Reviews table contained several good attributes but also some unusable attributes
related to the classification of a review as ``funny'', ``cool'', or ``useful''. We found
that conducting an initial data audit up front allowed us to more productively focus our
efforts.

\subsection{Replication of Yu et. al.}

They used four models, and we replicated three of them. Some of our results differed
from their findings.

\subsection{Extension of Yu et. al.}

We extended their approach by using a decision tree classifier.

\section{Code}

We used scikit-learn. We also used some text stuff because it was in the replication
paper.


\section{Results}

Reviewing results here


\section{Discussion}

We found that a certain model worked better. We found certain patterns
in the data, etc. We also have certain caveats related to the data that
we'll mention here.

\section{Future Work}

We may make a recommendation API for Yelp.

\section{Conclusion}

We found some things that worked well and other things that worked
less well. There seems to be a positive direction forward taking a
certain approach.


\bibliography{references} 
\bibliographystyle{ieeetr}

\end{document}
